\section{Introduction}

The paper attempts to detail the evolution of a legacy software that was taken over for four months. The following segments contain information about how the new system is designed, the process applied in making it, and key points learned.

First, we present diagrams from different perspectives that help understanding the interactions of important system constituents and how they are allocated on deployment environments. Next, direct dependencies are listed along with their description and date of addition to the project. We provide two ways of initiating the system with the usage of a virtual machine environment management tool and infrastructure as code. The latter of the first section contains some metrics and observations about the current state of the system, and how we have come to the conclusion of selecting our licenses.

Second, we detail the team's way of collaborating, the organization of the version control system and how the process relates to the Three Ways of working. Additionally, security assessment results are provided, and how monitoring and logging contributes to inspectable system telemetries. The last sections contain a thorough overview of the stages used in the CI/CD pipeline, how scaling is possible, as well as a future outlook of how these aspects can be improved.

Finally, the last segment contains key items that proved to have significant impact on the delivery of the project, or aspects that might be vital to remember in the future.