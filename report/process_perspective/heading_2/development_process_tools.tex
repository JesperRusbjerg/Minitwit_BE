\subsection{Applied Development Process and Tools}

%Our process was in a form of iterative agile software development without any specific name. We adopted overall agile principles and made them work for our use case. When it comes to the workflow a Trello (Kanban) board was a central point of it. It was divided into 4 major sections: Backlog, Doing, In-review, and done. Later the done section evolved and was split, having a separate section for each week to give a better overview of completed tasks. Regarding task derivation, refinement/alignment, and assignment, we had a dedicated timeslot for that each Tuesday after the DevOps class. During these sessions, we made sure that all of us knew the scope of work, we understood exactly what the tasks entail, and that everyone had at least one responsibility until the next time. As previously mentioned, we would often work in a group of two, providing each other with continuous feedback and learnings. The following week, we would pitch our work on the weekly standups, explaining it to everybody and collecting feedback for possible improvements. If any of them were proposed, we would reiterate the task and finish it by Sunday. It was when we would usually make releases and deploy new changes to “production”, following a procedure described in “Deployment pipelines”.

Our process was in a form of iterative agile software development (1 week sprint duration). We adopted “The three ways of working” initially used both in The Phoenix Project and The DevOps Handbook, as well as overall agile principles and tailored them for the needs of our project~\cite{devOpsHandbook}.

When it comes to the \textbf{\textit{'1. Flow'}}, Trello (Kanban) board was a central point of it. It was divided into 4 major sections: Backlog, Doing, In-review, and done (later divided into weeks). Regarding task derivation, refinement/alignment, and assignment, we had a dedicated timeslot for that each Tuesday after the DevOps class. During these sessions, we made sure that all of us knew the scope of work, we understood exactly what the tasks entail, and that everyone had at least one responsibility until the next time. 

We would often work in a group of two, fostering \textbf{\textit{'3. Continuous learning and Experimentation'}}. Problems were discussed and tackled together by the group as everyone represented a growth mindset. It was important for us to continuously push and improve upon the infrastructure while challenging ourselves. Moreover, for the experimentation part, we have treated the frontend with similar practices before applying it to backend - as an initial sandbox. This settled ground understanding of the exploited technology and revealed few issues prior risking backend crushing down.

The following week, we would pitch our work on the weekly standups, explaining it to everybody and collecting \textbf{\textit{'2. Feedback'}} for possible improvements, complementary to pull request reviews and automatic CI pipeline tests/scans. If any of them were proposed, we would reiterate the task and finish it by Sunday. It was when we would usually make releases and deploy new changes to “production”, following a procedure described in “\nameref{cicd:label}”.
