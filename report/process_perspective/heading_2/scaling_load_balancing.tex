\subsection{Scaling, Load Balancing and Redeployment}

The system is built using the containerization tool "Docker" we had initially created a docker-compose file~\cite{dockerComposeDefault} which can spin up all of our services on the server. In order to support scaling, the team created a swarm cluster by adding two new droplets. The original server, which has the most significant amount of resources at its disposal, became the Manager + Leader of the swarm. This service hosts an instance of the backend, the database, and logging/monitoring. The two new servers, the "workers", are running an instance of the backend. In order to create the swarm, we turned our docker-compose file into a docker-stack file~\cite{dockerCompose}. We are currently using docker-swarms standard load balancing. However, it is possible to implement an external load balancer such as a reverse proxy Nginx service, should the need for more control of load balancing arise. In terms of redeployment: Docker Swarm applies the Rolling deployment strategy by default; implementing canary or blue-green is possible but requires additional setup~\cite{colorfulDeployments}.