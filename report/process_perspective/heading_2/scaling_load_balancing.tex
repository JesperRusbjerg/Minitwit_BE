\subsection{Scaling and Load Balancing and Redployment}

The system is built using the containerization tool "Docker" we had initially created a docker-compose file which can spin up all of our services on the server. In order to support scaling, the team created a swarm cluster by adding two new droplets to the server. The original server, which has the most significant amount of resources at its disposal, became the Manager+Leader of the swarm. This service hosts an instance of the backend, the database, and logging/monitoring. The two new servers, the "workers", are running an instance of the backend. In order to create the swarm, we turned our docker-compose file into a docker-stack file[GITHUB LINK  https://github.com/JesperRusbjerg/minitwit\_BE/blob/main/Minitwit\_BE/docker-swarm-compose.yml (Currently an empty file, but I will fix it, just link to that file)]. We are currently using docker-swarms standard load balancing. However, it is possible to implement an external load balancer such as a reverse proxy Nginx service, should the need for more control of load balancing arise. In terms of redeployment: Dockerswarm applies the Rolling deployment strategy by default; implementing canary or blue-green is possible but requires additional setup[ADD REFERNCE TO BIB AND REF IT HERE: https://opensource.com/article/17/5/colorful-deployments].