\subsection*{Three ways of working}
\addcontentsline{toc}{subsection}{Three ways of working}

The Three Ways was initially coined both in The Phoenix Project and The DevOps Handbook. The notion consists of three parts.

\subsection*{Flow}
We adhered to the first principle from the beginning of the project by utilizing kanban boards to make our work more visible. We deemed that using a simple Trello board for this project was sufficient to store our backlog, WIP, and completed tasks. Each Tuesday after class, we would conduct knowledge sharing sessions, provide each other with feedback, and then collect the following week's tasks by discussing the upcoming implementations. Each member was committed to completing their responsible tasks before the following Tuesday, serving as the commitment to the team's progress. Limiting the number of work in progress items did not seem necessary as we did not have an overwhelming number of dependent and blocking relationships between tasks. This might have resulted from the reduction in the sizes of the work items. We tried focusing on what was necessary to get done and "gold plated" the project if sufficient time was available. Each member had an accountability area, and this was assigned based on the member's proficiency in a certain section of the project. This resulted in less context switching and faster development time.
\subsection*{Feedback}

Receiving constant feedback on our work allowed us to avoid and mitigate problems closer to the source when produced. There were multiple sources of continuous feedback: Receiving feedback from other group members every week, pull request reviews and the CI pipeline checking for static analysis errors, unit tests failing/passing, and docker scan scanning for security vulnerabilities in the images. The above has contributed significantly to increased productivity, fewer errors and ensuring that no error-prone code made it to production. In addition to the feedback loop we created ourselves, the simulator also acted as a feedback loop; every 3 hours, the team had an update on whether or not the system fulfilled the simulation requirements.


\subsection*{Continual Learning and Experimentation}
The team adhered to the third principle by fostering an inclusive and non-judgmental working environment. We had a complete understanding that we are doing work under safe circumstances and that making a mistake is another learning opportunity to improve upon our DevOps skills. Changes to the codebase have taken place without fear of failure, and the mistakes pointed out served as an enhancing activity, not a destructive one. Fortunately, everyone represented a growth mindset. Therefore problems were discussed openly. It was important to continuously push ourselves to better and improve upon the infrastructure while challenging ourselves.